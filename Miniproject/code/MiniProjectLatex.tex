\documentclass[11pt]{article}

\title{A Simple Document}

\author{Your Name}

\date{}

\begin{document}
  \maketitle
  
  \begin{abstract}
    This paper analyzes a seminal equation in population biology.
  \end{abstract}
  
  \section{Introduction}
    The fundamental nature of biology is complex. As increasingly vast amounts of sophisticated data regarding the biological world becomes more readily available through the evolution of technology, more complex mechanisms and tools are required to explain biological phenomena (Bolker et al., 2013). Null hypothesis-testing becomes ever more obsolete as researchers adopt more intricate and robust approaches to exploring data (Johnson et al., 2004). Model selection, based upon likelihood theory, is quickly becoming the most popular method in certain fields, such as microbiology (Ferrer et al., 2009). This approach allows for the handling of irregular, nonlinear, heteroscedastic data that was unsuitable for the simple ANOVA analyses of the past (Bolker et al., 2013). Biology is becoming an increasingly inter-disciplinary field as the incorporation of mathematics, computer science and physics becomes more accepted (Gunawardena, 2014). Many biological phenomena can be fitted to differential equations, and in this way applied mathematics gives us the tools, via mathematical models, to understand complexity (Transtrum et al., 2016). Mathematical models can not solely express the complexity of a biological system, however they provide a good representation when they are intricate enough to represent the system, while being transparent enough to allow for practical inferences and predictions across variable conditions (Transtrum et al., 2016). Model selection has three main advantages over null hypothesis-testing. It allows for multiple rival models to be evaluated against a data set simultaneously, instead of against an arbitrary probability threshold. The degree to which a model fits the data, the more support it lends to the associated hypothesis. Additionally, weight-based rankings provide a quantitative method of measuring support to said hypothesis. Finally, in cases where models have indistinguishable or nearly equal AIC (Akaike Information Criterion) values, model averaging allows us to combat this problem by ensuring robust parameter estimates and predictions (Johnson et al., 2004). Mathematical models have had a significant impact on furthering our understanding of experimental observations, as well as biological systems and mechanisms, for the last twenty years (Jin, 2017). 
  
  \section{Materials \& Methods}
  
  A foundational equation of population biology is:
  
  \begin{equation}
    \frac{dN}{dt} = r N (1 - \frac{N}{K})
  \end{equation}
  
  It was first proposed by Verhulst in 1838 \cite{verhulst1838notice}.
  
  \bibliographystyle{apalike}
  
  \bibliography{Biblio1}

\end{document}